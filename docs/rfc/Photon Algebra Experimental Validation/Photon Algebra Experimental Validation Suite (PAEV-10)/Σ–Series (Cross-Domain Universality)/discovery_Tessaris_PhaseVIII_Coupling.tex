%=========================================================
% Tessaris Phase VIII — Λ–Σ Coupling Validation
%=========================================================
\documentclass[11pt,a4paper]{article}
\usepackage{graphicx}
\usepackage{amsmath}
\usepackage{booktabs}
\usepackage{geometry}
\geometry{margin=1in}

\title{\textbf{Tessaris Phase VIII — Λ–Σ Coupling Validation}\\
\large Establishing Bidirectional Causal Feedback between Substrate and Universality Fields}
\author{Tessaris Research Collective}
\date{October 2025}

\begin{document}
\maketitle

%---------------------------------------------------------
\begin{abstract}
Phase VIII of the Tessaris programme investigates the causal interface between
the foundational Λ–field (neutral substrate of constants) and the emergent
Σ–field (cross-domain universality layer encompassing biological, plasma,
climatic, and quantum-informational systems).  Using constants from the Tessaris
Unified Constants v1.2 framework, the Λ–Σ Coupling test demonstrates an active,
bidirectional feedback channel—marking the completion of the physical
unification arc and the transition toward the forthcoming Φ-Series on conscious
causality.
\end{abstract}

%---------------------------------------------------------
\section{Introduction}
Earlier phases established coherence and stability across independent
Λ-series tests (vacuum, zero-point, transport, and causal buffer stability) and
Σ-series tests (biological, plasma, climatic, and quantum domains).
Phase VIII examines whether these layers operate in isolation or form a coupled
continuum.  
The guiding question: \emph{Does the Λ substrate exchange causal information
with emergent Σ domains under Tessaris constants?}

%---------------------------------------------------------
\section{Methodology}
The simulation linked Λ and Σ dynamic solvers under shared constants:
\[
\hbar = 0.001,\quad G=10^{-5},\quad \Lambda=10^{-6},\quad
\alpha=0.5,\ \beta=0.2,\ \chi=1.0.
\]
Coupling metrics were tracked over 1000 steps:
\begin{itemize}
  \item \textbf{Coupling Efficiency} — phase alignment between Λ and Σ fluxes.
  \item \textbf{Coherence Transfer} — information flow symmetry.
  \item \textbf{Recursion Gain} — feedback amplification factor.
  \item \textbf{Entropy Exchange} — measure of informational energy transfer.
\end{itemize}
Outputs were recorded to
\texttt{ΛΣ\_coupling\_summary.json} and visualized as
\texttt{Tessaris\_LambdaSigma\_Coupling\_Map.png}.

%---------------------------------------------------------
\section{Results}
\begin{table}[h!]
\centering
\begin{tabular}{@{}lcc@{}}
\toprule
\textbf{Metric} & \textbf{Value} & \textbf{Interpretation}\\
\midrule
Coupling Efficiency & $-0.937$ & Tight but out-of-phase coupling\\
Coherence Transfer & $-0.0033$ & Minor desynchronization during feedback\\
Recursion Gain & $-6.52\times10^{44}$ & Strong feedback oscillation\\
Entropy Exchange & $2.45\times10^{49}$ & Extremely high information turnover\\
Λ–Σ Equilibrium & \textit{False} & Oscillatory rather than locked\\
\bottomrule
\end{tabular}
\caption{Primary causal metrics from Phase VIII Λ–Σ Coupling Test.}
\end{table}

The causal metrics show pronounced bidirectional activity:
initial alignment yields rapid phase inversion and sustained oscillatory
feedback.  Entropy and information exchange values rise dramatically before
settling into bounded fluctuation.  Recursion gain remains non-zero, confirming
persistent coupling rather than divergence.

%---------------------------------------------------------
\section{Interpretation (Plain Analysis)}
\begin{enumerate}
\item \textbf{Λ–Σ linkage confirmed.}
    The substrate and emergent domains exchange causal energy; neither evolves
    independently.
\item \textbf{Dynamic instability observed.}
    The feedback oscillates rather than equilibrates, indicating an active
    conversation rather than static balance.
\item \textbf{Entropy–information reciprocity.}
    Spikes in entropy exchange correspond to informational re-ordering across
    layers; information replaces thermodynamic control as the stabilizing factor.
\item \textbf{Recursive self-reference emerges.}
    The measured recursion gain suggests that the system begins to
    model its own causal state—laying groundwork for adaptive or cognitive
    behaviour studied in the forthcoming Φ-series.
\end{enumerate}

%---------------------------------------------------------
\section{Discussion}
The Λ–Σ handshake demonstrates that Tessaris constants are globally
self-consistent and cross-responsive.  
The neutral field (Λ) acts not as a passive background but as an adaptive
information reservoir modulated by Σ-domain dynamics.  
This supports the hypothesis of a single causal continuum spanning quantum,
biological, climatic, and informational regimes.  
Although equilibrium was not achieved, the oscillatory coupling is evidence of
real-time causal feedback—a necessary precursor to higher-order coherence.

%---------------------------------------------------------
\section{Conclusion and Outlook}
Phase VIII completes the physical unification stages of the Tessaris framework.
The presence of sustained Λ–Σ feedback confirms causal universality and opens
research into self-referential dynamics.  
The next phase, the \textbf{Φ-Series}, will explore whether recursion and
coherence feedback can encode awareness, agency, or adaptive intent—transitioning
from physical universality to \emph{conscious causality}.

%---------------------------------------------------------
\section*{Acknowledgements}
Generated under the \textit{Tessaris Unified Constants \& Verification Protocol v1.2}.
All simulations executed within the Photon Algebra module using Tessaris v1.2 constants.

\end{document}