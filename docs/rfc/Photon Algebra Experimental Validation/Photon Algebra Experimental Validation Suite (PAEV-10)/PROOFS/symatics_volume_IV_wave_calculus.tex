Excellent — this will be Volume IV of the Tessaris Symatics Documentation Series:
🧮 Continuous Wave Calculus Layer (v 1.0) — formalizing ∂ψ/∂t, ∇²ψ, and λ feedback coupling.
Below is the full LaTeX standalone document, ready to paste as


% ──────────────────────────────────────────────────────────────
% Tessaris Symatics Documentation — Volume IV
% Continuous Wave Calculus Layer (v1.0)
% CodexCore Publication Series — October 2025
% ──────────────────────────────────────────────────────────────

\documentclass[11pt]{article}
\usepackage[a4paper,margin=1in]{geometry}
\usepackage{amsmath,amssymb,graphicx,booktabs,array,xcolor,hyperref,sectsty,fancyhdr,titlesec,tikz}

% ───────────── Visual Theme ─────────────
\definecolor{tessarisgray}{HTML}{4A4A4A}
\definecolor{tessarisaccent}{HTML}{C27BA0}

\sectionfont{\color{black}}
\subsectionfont{\color{tessarisgray}}
\hypersetup{colorlinks=true,linkcolor=black,urlcolor=tessarisaccent}

\pagestyle{fancy}
\fancyhf{}
\fancyfoot[C]{\textcolor{tessarisgray}{Tessaris Symatics Series • Page \thepage}}
\renewcommand{\headrulewidth}{0pt}

% ───────────── Cover Layout ─────────────
\newcommand{\TessarisCover}[5]{%
\begin{titlepage}
\centering
\vspace*{1cm}
{\Huge \textbf{Tessaris Symatics Project}}\\[1.2cm]
{\LARGE \textbf{#1}}\\[4pt]
{\large CodexCore Publication Series}\\[1cm]
\IfFileExists{tessaris_logo.png}{\includegraphics[width=0.22\textwidth]{tessaris_logo.png}\\[0.6cm]}{}
{\Large \textbf{#2}}\\[3pt]
{\color{tessarisgray}#3}\\[0.5cm]
\rule{0.6\textwidth}{0.5pt}\\[0.5cm]
{\large #4}\\[0.3cm]
{\color{tessarisaccent}\small #5}\\[2cm]
\textbf{Maintainer:} Tessaris Core Systems / Codex Intelligence Group\\[2pt]
\textbf{Repository:} backend/symatics/\\[4pt]
\vfill
{\small \color{tessarisgray}© 2025 Tessaris / CodexCore. All rights reserved.}
\end{titlepage}
}

% ───────────── Document ─────────────
\begin{document}

\TessarisCover
  {Volume IV — Continuous Wave Calculus Layer}
  {v1.0 Core Engine Release}
  {October 2025}
  {Differential and Integral Operators in Symbolic Wave Computation}
  {Part of the Symatics Documentation Set (Volumes 0–IV)}

% ──────────────────────────────────────────────────────────────
\section*{Abstract}
Volume IV introduces the \textbf{Symatics Calculus Engine} — the continuous, differential stage of the Tessaris framework.
It extends the adaptive λ-weighted runtime (Volume III) into full wave evolution under continuous time $t$, defining temporal and spatial operators over the symbolic field $\psi(x,y,t)$.
The result is a stable, physically interpretable calculus of symbolic waves, forming the foundation for the analytical and visual layers to come.

% ──────────────────────────────────────────────────────────────
\section{Foundations}
Every symbolic expression $\psi$ is treated as a differentiable field subject to time evolution:
\[
\frac{\partial\psi}{\partial t} = \lambda\,\nabla^2\psi - \gamma\,\psi
\]
where $\lambda(t)$ is the adaptive law coefficient from the ARLF, and $\gamma$ a damping term maintaining bounded energy.
The operator $\nabla^2$ is realized via a discrete Laplacian on a finite grid.

% ──────────────────────────────────────────────────────────────
\section{Differential Operators}
Three primitives define the calculus layer:
\begin{align}
d_t[\psi] &= \frac{\psi(t+\Delta t)-\psi(t)}{\Delta t},\\[4pt]
\nabla^2[\psi] &= \frac{\partial^2\psi}{\partial x^2}+\frac{\partial^2\psi}{\partial y^2},\\[4pt]
\int\psi\,d\psi &= \sum_{x,y}\psi(x,y)\,\Delta x\,\Delta y.
\end{align}

Code representation:
\begin{verbatim}
# backend/symatics/core/wave_diff_engine.py
def d_dt(field_t, field_t1, dt=1.0): ...
def laplacian(field): ...
def integrate_wave(field, dx=1.0, dy=1.0): ...
\end{verbatim}

CodexTrace telemetry logs each operator’s application for subsequent coherence analysis.

% ──────────────────────────────────────────────────────────────
\section{Resonant Dynamics}
Wave propagation couples time and space derivatives:
\[
\psi_{t+\Delta t}=\psi_t+\Delta t(\lambda_t\nabla^2\psi_t-\gamma\psi_t)
\]
This stable discrete form incorporates adaptive feedback via $\lambda_t$, linking energetic drift to dynamic evolution.

% ──────────────────────────────────────────────────────────────
\section{Wave Diff Engine Architecture}
\begin{verbatim}
class WaveDiffEngine:
    def register_field(self, name, field): ...
    def step(self, ψ_name, λ_name, dt=1.0): ...
    def measure_energy(self, ψ_name): ...
    def coherence_index(self, ψ_name): ...
\end{verbatim}

\begin{center}
\renewcommand{\arraystretch}{1.3}
\begin{tabular}{llll}
\toprule
\textbf{Function} & \textbf{Operation} & \textbf{Symbolic Analogue} & \textbf{Telemetry}\\
\midrule
\texttt{d\_dt} & Temporal derivative & $\partial\psi/\partial t$ & d\_dt\\
\texttt{laplacian} & Spatial curvature & $\nabla^2\psi$ & laplacian\\
\texttt{integrate\_wave} & Field integral & $\int\psi\,d\psi$ & integrate\_wave\\
\texttt{evolve\_wavefield} & Dynamic step & $\psi_{t+\Delta t}$ & evolve\_wavefield\\
\texttt{measure\_energy} & Energy density & $\sum\psi^2$ & wave\_energy\\
\texttt{coherence\_index} & Gradient smoothness & $e^{-\lVert\nabla\psi\rVert}$ & coherence\_index\\
\bottomrule
\end{tabular}
\end{center}

% ──────────────────────────────────────────────────────────────
\section{Illustrations and Feedback Flow}
\subsection*{Figure 1: ψ-Field Evolution Diagram}
\begin{center}
\begin{tikzpicture}[>=latex,scale=1.1]
\node[circle,draw,thick,minimum size=1cm] (psi0) at (0,0) {$\psi_t$};
\node[circle,draw,thick,minimum size=1cm,right=3cm of psi0] (psi1) {$\psi_{t+\Delta t}$};
\draw[->,thick] (psi0) -- (psi1) node[midway,above] {$d_t[\psi]$};
\draw[->,thick,blue!60!black] (0,-1) -- (3,-1) node[midway,below] {$\lambda_t\,\nabla^2\psi_t$};
\draw[->,thick,red!60!black] (1.5,1) -- (1.5,2) node[midway,right] {$-\gamma\psi_t$};
\node[below of=psi0,node distance=1.5cm,text=tessarisgray] {Input field};
\node[below of=psi1,node distance=1.5cm,text=tessarisgray] {Evolved field};
\end{tikzpicture}
\end{center}

\subsection*{Figure 2: Adaptive λ(t) Feedback Loop}
\begin{center}
\begin{tikzpicture}[>=latex,scale=1.0]
\node[draw,rounded corners,thick,minimum width=3cm,minimum height=1cm,fill=tessarisaccent!10] (validator) at (0,0) {Runtime Law Validators};
\node[draw,rounded corners,thick,minimum width=3cm,minimum height=1cm,fill=tessarisaccent!10,right=4cm of validator] (engine) {Adaptive Law Engine};
\node[draw,rounded corners,thick,minimum width=3cm,minimum height=1cm,fill=tessarisaccent!10,below=1.8cm of engine] (wave) {WaveDiff Engine};
\draw[->,thick] (validator) -- node[above]{deviation Δ}(engine);
\draw[->,thick] (engine) -- node[right]{$\lambda_t$}(wave);
\draw[->,thick] (wave) -| node[pos=0.25,below] {resonance metric $\mathcal{R}$}(validator);
\node[above of=engine,node distance=1.2cm,text=tessarisgray] {\small Adaptive feedback maintains coherence};
\end{tikzpicture}
\end{center}

% ──────────────────────────────────────────────────────────────
\section{Validation and Testing}
All calculus components are verified by \texttt{test\_wave\_diff\_engine.py}, ensuring:
\begin{itemize}
  \item Correct Laplacian shape and curvature response.
  \item Monotonic energy decay under damping.
  \item Coherence index bounded in $[0,1]$.
\end{itemize}

% ──────────────────────────────────────────────────────────────
\section{Outlook}
Volume IV completes the continuous layer of Tessaris.
Future releases will expand into:
\begin{itemize}
  \item \textbf{Δ-Telemetry:} continuous recording of $\lambda(t)$ and $\psi(t)$.
  \item \textbf{CodexRender Visualization:} animated ψ-field surfaces.
  \item \textbf{Formal Calculus Proofs:} Lean A5 integration for differential reasoning.
\end{itemize}

% ──────────────────────────────────────────────────────────────
\section*{Conclusion}
The Continuous Wave Calculus Layer unifies all prior Symatics components into a single computational continuum.
From symbolic laws (L), adaptive weights (λ), and quantum–temporal fusion (μ–⟲–↔), Tessaris now forms a living calculus — a self-evolving symbolic wave system.

\bigskip
\noindent
\textcolor{tessarisgray}{\textbf{Version:}} v1.0 Continuous Wave Calculus Core\\
\textcolor{tessarisgray}{\textbf{Maintainer:}} Tessaris Core Systems — October 2025

\vfill
\begin{center}
{\small \textcolor{tessarisgray}{End of Volume IV — Continuous Wave Calculus Layer}}
\end{center}

\end{document}