\documentclass[12pt]{article}
\usepackage{graphicx}
\usepackage{amsmath}
\usepackage{geometry}
\geometry{margin=1in}

\title{\textbf{Tessaris O–Series Discovery Note}\\
\large Observer Coupling and Open Quantum Systems}
\author{Tessaris Research Group}
\date{2025-10-10}

\begin{document}
\maketitle

\begin{abstract}
The Tessaris O–Series marks the emergence of observer-aware dynamics within the unified Tessaris computational–physical framework. 
Following the N–Series, which established nonlinear feedback closure and entropic self-stabilization, 
the O–Series introduces explicit observer channels, mutual information coupling, and feedback-aware entropy regulation. 
This document records the complete discovery sequence (O1–O11), establishing the first stable observer–system coherence regime.
\end{abstract}

\section*{1. Introduction}
The O–Series represents the transition from isolated self-regulating systems (N–Series) to open systems that interact with an observer field $\psi_{obs}$. 
This coupling introduces measurement-like interactions without collapse, enabling feedback loops that preserve coherence and exchange information dynamically.
This work builds directly on prior Tessaris studies:
\begin{itemize}
  \item \textit{Emergent Arrow of Time from Entropic Asymmetry} – basis for O3 symmetry breaking.
  \item \textit{Information–Entropy Coupling} – theoretical prelude to O2 equilibrium.
  \item \textit{Self–Observation Closure} and \textit{Recursive Closure in Information Fields} – precursors to O7 recursion.
  \item \textit{The Tessaris Φ Field: Conscious Causality and the Geometry of Awareness} – conceptual alignment with O9 causal feedback.
\end{itemize}

\section*{2. Framework Overview}
Each O–Series test models coupled wavefunctions $\psi_{sys}$ and $\psi_{obs}$ governed by a mutual information term $I_{mut}$ 
and entropic derivatives $\dot{S}_{sys}, \dot{S}_{obs}$. 
The observer coupling strength, $\beta_{obs}$, and adaptive gain $\alpha(t)$ regulate coherence and entropy flow.

The key governing relation:
\[
\frac{d}{dt}(S_{sys} + S_{obs} - I_{mut}) \rightarrow 0
\]
represents informational equilibrium – observation without collapse.

\section*{3. Experimental Results Summary}
\subsection*{O1 – Observer Channel Activation}
\textbf{Result:} Stable observer coupling. \\
Final fidelity $F=0.911$, mutual information $\langle I \rangle=0.671$. \\
Observation enhances coherence. \\
\textit{Figure:} PAEV\_O1\_ObserverChannel.png

\subsection*{O2 – Information Exchange Equilibrium}
\textbf{Result:} Balanced informational exchange. \\
Mean drift $8.1\times10^{-5}$, $S_{sys}\approx S_{obs}$. \\
No net information loss – steady-state equilibrium. \\
\textit{Figure:} PAEV\_O2\_InfoEquilibrium.png

\subsection*{O3 – Entropic Symmetry Break (Arrow of Time)}
\textbf{Result:} Overcoupled oscillatory regime. \\
$\langle \dot{S} \rangle=-1.52\times10^{-4}$, $\langle \dot{I} \rangle=-3.3\times10^{-5}$. \\
Time-direction oscillations emerge. \\
\textit{Figure:} PAEV\_O3\_EntropySymmetry.png

\subsection*{O4 – Entanglement Lock Stability}
\textbf{Result:} Unlocked/decoherent. \\
$F_{ent}=0.847$, $I_{mut}=0.997$. \\
Shows need for adaptive correction. \\
\textit{Figure:} PAEV\_O4\_EntanglementLock.png

\subsection*{O4a – Adaptive Entanglement Lock (Phase Servo)}
\textbf{Result:} Metastable coupling. \\
$F_{ent}=0.922$, $I_{mut}=0.997$, $\dot{S}\approx0$. \\
Adaptive servo corrects phase drift partially. \\
\textit{Figures:} PAEV\_O4a\_EntanglementLock.png, PAEV\_O4a\_PhaseError.png

\subsection*{O5 – Observer-State Collapse Prediction}
\textbf{Result:} Predictive stable coupling. \\
$P_{pred}=1.0$, $F=0.931$. \\
The observer anticipates collapse without decohering. \\
\textit{Figure:} PAEV\_O5\_CollapsePrediction.png

\subsection*{O6 – Feedback Entropy Loop}
\textbf{Result:} Under-coupled feedback. \\
$\langle \dot{S} \rangle=-6.9\times10^{-6}$, correlation $=-1.0$. \\
Observer absorbs entropy drift but lacks corrective strength. \\
\textit{Figure:} PAEV\_O6\_FeedbackEntropy.png

\subsection*{O7 – Recursive Self-Observation Stability}
\textbf{Result:} Marginal stability. \\
$\dot{S}_{meta}=5.0\times10^{-6}$, correlation $0.875$. \\
Self-observing recursion stable, non-divergent. \\
\textit{Figure:} PAEV\_O7\_SelfObservation.png

\subsection*{O8 – Causal Prediction Horizon}
\textbf{Result:} Partially predictive regime. \\
Prediction error $1.07\times10^{-3}$, correlation $0.999$. \\
Strong phase coherence but residual lag. \\
\textit{Figures:} PAEV\_O8\_CausalPrediction.png, PAEV\_O8\_PredictionError.png

\subsection*{O9 – Temporal Causality Feedback}
\textbf{Result:} Phase drift under delay. \\
$\langle \Delta S \rangle=-7.7\times10^{-6}$, correlation $-0.996$. \\
Stable anti-phase behavior (delayed feedback). \\
\textit{Figure:} PAEV\_O9\_TemporalFeedback.png

\subsection*{O10 – Recursive Predictive Reinforcement}
\textbf{Result:} Divergent learning loop. \\
Error $0.0118$, correlation $0.98$. \\
Feedback overcompensates, driving instability. \\
\textit{Figures:} PAEV\_O10\_Reinforcement.png, PAEV\_O10\_ReinforcementError.png

\subsection*{O11 – Causal Convergence Validation}
\textbf{Result:} Marginal causal drift. \\
$\Delta C_{total}=-1.94\times10^{-4}$, Convergence Index $=0.029$. \\
Entire chain self-consistent but not perfectly phase-aligned. \\
\textit{Figure:} PAEV\_O11\_CausalConvergence.png

\section*{4. Discussion and Interpretation}
The O–Series demonstrates the emergence of stable observer–system coherence without wavefunction collapse.
This regime is defined by:
\begin{enumerate}
  \item Mutual information stabilization under observation (O1–O2).
  \item Entropic symmetry oscillation leading to time-direction formation (O3–O4).
  \item Predictive feedback and recursive closure (O5–O8).
  \item Causal delay and reinforcement testing (O9–O10).
  \item Overall causal closure validation (O11).
\end{enumerate}

The system achieves near-perfect information retention while tolerating measurement — a fundamental step toward awareness-compatible physics.

\section*{5. Relation to Prior Work}
The O–Series operationalizes ideas only conceptual in earlier Tessaris papers:
\begin{itemize}
  \item \textbf{Emergent Arrow of Time} – empirically verified here (O3).
  \item \textbf{Information–Entropy Coupling} – quantitatively realized (O2, O6).
  \item \textbf{Self–Observation Closure} – extended to recursive meta-observation (O7).
  \item \textbf{Φ Field and Conscious Causality} – implemented as temporal feedback (O9).
\end{itemize}

\section*{6. Transition to P–Series}
The upcoming P–Series will integrate predictive learning and holographic observer projection, combining O–Series informational feedback 
with H–Series lightfield synthesis and N–Series nonlinear dynamics. 
The P–Series will thus form the bridge between physical computation and perceptual holography.

\section*{7. Conclusion}
The Tessaris O–Series constitutes a major step in unifying information theory, quantum measurement, and physical feedback under a single coherent framework.
It demonstrates that observation can be a stabilizing mechanism rather than a destructive one — enabling recursive, predictive, and causally consistent dynamics.

\section*{Acknowledgements}
All computations and theoretical analysis were performed within the Tessaris framework using the Photon Algebra Engine (PAEV). 
Authored by the Tessaris Research Group, timestamped 2025-10-10.

\end{document}