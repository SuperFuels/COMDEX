\appendix
\section*{Appendix A: Analytical Assessment of the Tessaris Unified Equation (Post-G-Series)}

\subsection*{Overview}
This appendix provides a theoretical analysis of the post-G-series evolution of the Tessaris Unified Equation (TUE), which introduces cross-domain coupling and Lyapunov feedback as stabilizing mechanisms.  
It formalizes the interpretation of each extended term and establishes correspondence between the analytical formulation and its computational realization within the Tessaris simulation framework.

\subsection*{1. Core Identity}
\begin{equation}
E = \frac{c^4}{8 \pi G_{\text{eff}}} R_{\psi}
  + k_B T_{\text{eff}} S
  + \hbar \dot{I}_{\text{mut}}
  + \Lambda_{\text{eff}} \Phi_{\text{couple}} .
\end{equation}
\begin{itemize}
  \item $R_{\psi}$ – curvature of the coupled visible–hidden manifold, potentially encoding dark-sector or sub-quantum curvature effects.  
  \item $k_B T_{\text{eff}} S$ – thermodynamic energy, reflecting emergent entropy-temperature relations consistent with entropic gravity.  
  \item $\hbar \dot{I}_{\text{mut}}$ – informational flux, computationally modeled as $\mathrm{Re}(\dot{\psi}\,\psi^*)$, representing quantum coherence transfer.  
  \item $\Lambda_{\text{eff}} \Phi_{\text{couple}}$ – coupling potential connecting geometric and informational domains, where $\Phi_{\text{couple}}\!\sim\!\kappa\dot{I}_{\text{mut}}$ in the computational basis.
\end{itemize}

\subsection*{2. Dynamic Feedback Law}
\begin{equation}
\frac{dE}{dt} = -\kappa_L\nabla_{\text{sys}}V(E,S,I,R)
  + \gamma_{\text{sync}}\frac{d}{dt}(\Lambda_{\text{eff}}I_{\text{mut}}),
\end{equation}
ensuring Lyapunov-controlled descent along the system potential $V$.  
Here, $\kappa_L$ functions as a damping coefficient driving the system toward stability, while $\gamma_{\text{sync}}$ regulates phase coherence between curvature and information flux.  
This synchronization term establishes a feedback pathway through which informational coherence modulates the evolution of $\Lambda_{\text{eff}}$, potentially mirroring dark-energy adaptation.

\subsection*{3. Conservation with Cross-Domain Coupling}
\begin{equation}
\frac{d}{dt}\!\left(\frac{c^4}{8\pi G_{\text{eff}}}R_{\psi}
-\Lambda_{\text{eff}}\rho_c a^3
+\rho_{\text{fields}}+\rho_{\text{info}}+\rho_{\text{couple}}\right)=0,
\end{equation}
where $\rho_{\text{couple}}\!\propto\!\Lambda_{\text{eff}}\Phi_{\text{couple}}$ represents stored linkage energy at the information–geometry interface.  
This term ensures conservation across both physical and informational domains.

\subsection*{4. Effective Mass–Energy Equivalence}
\begin{equation}
M_{\text{eff}}c^2=\frac{c^4}{8\pi G_{\text{eff}}}R_{\psi}
+k_B T_{\text{eff}}S+\hbar\dot{I}_{\text{mut}}
+\Lambda_{\text{eff}}\Phi_{\text{couple}},
\end{equation}
extending Einstein’s relation to include geometric, thermodynamic, informational, and coupling contributions, yielding an emergent total mass–energy form.

\subsection*{5. Stability and Adaptive Feedback}
\begin{align}
\dot G_{\text{eff}} &= -k_G\,\frac{\partial E}{\partial R_{\psi}}, &
\dot\Lambda_{\text{eff}} &= -k_L\,\frac{\partial E}{\partial I_{\text{mut}}}, &
\dot T_{\text{eff}} &= -k_T\,\frac{\partial E}{\partial S}.
\end{align}
These adaptive couplings dynamically stabilize $E_{\text{tot}}$ through gradient descent, maintaining the Lyapunov lock condition ($E_{\text{stability}}\!\approx\!1$).  
The feedback loop continuously renormalizes the effective constants, enabling self-correcting evolution.

\subsection*{6. Unified Conservation Closure}
\begin{equation}
\Delta E_{\text{geom}}+\Delta E_{\text{matter}}
+\Delta E_{\text{info}}+\Delta E_{\text{couple}}=0,
\end{equation}
confirming total energy redistribution without loss—an informationally closed cosmological system.

\subsection*{7. Computational Implementation}
A simplified computational scheme illustrates the iterative feedback process:

\begin{verbatim}
for t in range(steps):
    R = mean(laplacian(kappa))
    S = -mean(|psi|^2 * log(|psi|^2))
    I_mut = mean(Re(psi_t * conj(psi)))
    E_tot = E_geom + E_therm + E_info + E_couple
    G_eff -= dt * k_G * ∂E/∂R
    Λ_eff -= dt * k_L * ∂E/∂I_mut
    T_eff -= dt * k_T * ∂E/∂S
\end{verbatim}

This pseudocode captures the dynamic equilibrium between curvature ($R$), entropy ($S$), and information flux ($I_{\text{mut}}$) under the influence of adaptive feedback.

\subsection*{8. Interpretation}
The Lyapunov-controlled closure mathematically defines a steady oscillatory attractor—a limit cycle where curvature, entropy, and information remain phase-locked.  
This generates a reproducible, testable model of informational cosmology, where stability directly maps to coherence in the energy–information continuum.

\subsection*{9. Assessment Summary}
The post-G-series TUE addresses earlier theoretical limitations by:
\begin{itemize}
  \item explicitly coupling $\dot{I}_{\text{mut}}$ to measurable coherence functions;
  \item dynamically renormalizing effective constants through adaptive feedback;
  \item enabling falsifiable predictions for curvature–information interactions.
\end{itemize}
The revised equation thus constitutes a rigorous, reproducible, and self-consistent foundation for the Tessaris Framework’s unified treatment of geometry, energy, and cognition.