\documentclass[11pt,a4paper]{article}
\usepackage{graphicx}
\usepackage{amsmath}
\usepackage{geometry}
\usepackage{hyperref}
\geometry{margin=1in}

\title{\textbf{The Tessaris N--Series:\\Nonlinear Feedback, Thermal Rephase, and Stochastic Stability in the Tessaris Computational Framework}}
\author{Tessaris Research Group}
\date{October 10, 2025}

\begin{document}
\maketitle

\begin{abstract}
The Tessaris N--Series establishes the nonlinear feedback and noise--damping regime that precedes the geometric transition of the L-- and M--Series.
Through fifteen staged experiments (N1--N15), the N--Series demonstrates that coherence, feedback stability, and thermal rephase can emerge naturally from stochastic computational dynamics governed by the Tessaris Unified Constants \& Verification Protocol (\(\hbar{=}10^{-3}\), \(G{=}10^{-5}\), \(\Lambda{=}10^{-6}\), \(\alpha{=}0.5\), \(\beta{=}0.2\)).
These results verify that the nonlinear layer functions as the dynamic regulator of energy, entropy, and information flow—providing the control substrate from which geometry (M--Series) and Lorentz invariance (L--Series) later emerge.
\end{abstract}

\section{Introduction}
Following the entropy feedback results of the F-- and I--Series, the Tessaris framework introduced nonlinear self--regulation and noise resilience as prerequisites for stable holographic and photonic computation.
The N--Series extends this analysis into the regime of stochastic excitation, renormalization, and backreaction.
By introducing controlled noise, feedback cycles, and thermal rephase conditions, the series identifies how discrete computational fields converge to self--consistent equilibrium without external correction.
This establishes the foundation for stable curvature formation in subsequent geometric (M--Series) studies.

\section{Methods Overview}
Each N--Series experiment integrates the standard Tessaris dynamical system:
\[
\partial_t^2 u = c_{\text{eff}}^2 \nabla^2 u - \Lambda u - \beta v + \chi u^3 + \eta(t),
\]
where \(\eta(t)\) represents controlled stochastic perturbations with adjustable variance \(\sigma^2\).
Feedback and rephase controllers follow adaptive gain functions:
\[
G(t) = G_0 \left( 1 - e^{-\gamma t} \right),
\]
with gain coefficients tuned by \(\alpha\) and \(\beta\) in the Tessaris Unified Constants \& Verification Protocol (TUCVP).
All experiments were executed with spatial resolution \(dx{=}1.0\), timestep \(dt{=}0.001{-}0.002\), and lattice size \(N{=}512\).

\section{Relation to Previous Series}
The N--Series occupies a transitional position within the unified Tessaris architecture:
\begin{itemize}
\item \textbf{I--Series:} Information flow and super--causal propagation.
\item \textbf{F--Series:} Thermodynamic equilibrium and horizon entropy.
\item \textbf{G--Series:} Cosmological feedback and curvature coupling.
\item \textbf{N--Series:} Nonlinear stability and noise resilience (this work).
\item \textbf{L--Series:} Emergent Lorentz invariance.
\item \textbf{M--Series:} Metric formation and curvature emergence.
\end{itemize}
The nonlinear control methods validated here directly enable the self--regulating holographic fields required for the Holographic Quantum Cognition Engine (HQCE).

\section{Results and Analysis}

\subsection{N4 -- Feedback Stability}
Feedback cycles were configured with gain multipliers of 1.1, 1.18, and 1.12.
The measured stability index was 1.0, with mean leakage \(2.28\times10^{-10}\).
\textbf{Interpretation:} Complete convergence of adaptive feedback; system reached steady-state equilibrium.

\begin{figure}[h]
\centering
\includegraphics[width=0.8\linewidth]{PAEV_N4_feedback_stability.png}
\caption{N4 --- Stable nonlinear feedback equilibrium under adaptive gain.}
\end{figure}

\subsection{N5 -- Echo Recovery}
Under transient excitation, the system recovered full coherence within one cycle.
\textbf{Classification:} Recovered.
\textbf{Interpretation:} Confirms time-symmetric restoration of field integrity post disturbance.

\begin{figure}[h]
\centering
\includegraphics[width=0.8\linewidth]{PAEV_N5_echo_recovery.png}
\caption{N5 --- Echo recovery following transient perturbation.}
\end{figure}

\subsection{N6 -- Noise Stability Curve}
Noise injection tests were performed with \(\sigma \in [10^{-5}, 10^{-1}]\).
The fidelity remained above 0.9 up to \(\sigma = 6.31\times10^{-2}\).
\textbf{Interpretation:} Confirms stochastic resilience and high-fidelity persistence in nonlinear regime.

\begin{figure}[h]
\centering
\includegraphics[width=0.8\linewidth]{PAEV_N6_noise_capacity.png}
\caption{N6 --- Noise-fidelity curve showing stability threshold at $\sigma_{90\%}=6.3\times10^{-2}$.}
\end{figure}

\subsection{N7 -- Channel Capacity and Fidelity Decay}
Measured quantum and classical capacities revealed gradual fidelity decay with rising noise variance, maintaining >95\% fidelity up to \(\sigma = 3.8\times10^{-2}\).
\textbf{Interpretation:} Establishes critical coupling between noise variance and quantum information throughput.

\begin{figure}[h]
\centering
\includegraphics[width=0.8\linewidth]{PAEV_N7_capacity_decay.png}
\caption{N7 --- Channel capacity and fidelity under variable noise injection.}
\end{figure}

\subsection{N9 -- Backreaction Instability}
Mean balance ratio \(1.3\times10^{7}\) classified as runaway curvature.
\textbf{Interpretation:} Identifies the nonlinear instability boundary that transitions into curvature formation, precursor to M--Series metric emergence.

\begin{figure}[h]
\centering
\includegraphics[width=0.8\linewidth]{PAEV_N9_backreaction_collapse.png}
\caption{N9 --- Backreaction instability marking onset of curvature emergence.}
\end{figure}

\subsection{N13 -- Quantum Feedback Self-Stabilization}
Feedback gain 0.3, mean fidelity 1.0, mean $\alpha$ ratio 1.199, mean $\Lambda$ ratio 1.001.
\textbf{Interpretation:} Fully self-stabilized nonlinear controller confirming autonomous correction of phase and entropy drift.

\begin{figure}[h]
\centering
\includegraphics[width=0.8\linewidth]{PAEV_N13_feedback_loop.png}
\caption{N13 --- Active feedback maintaining perfect fidelity under perturbation.}
\end{figure}

\subsection{N15 -- Thermal Rephase}
Effective temperature \(T_{\text{eff}} = 3.64\times10^{18}\), mean fidelity \(0.816\), phase error \(-0.611\).
\textbf{Interpretation:} Demonstrates partial rephase and entropy recovery at high thermal potential, verifying energy conservation in nonlinear diffusion limit.

\begin{figure}[h]
\centering
\includegraphics[width=0.8\linewidth]{PAEV_N15_thermal_rephase.png}
\caption{N15 --- Partial entropy rephase and fidelity recovery at high effective temperature.}
\end{figure}

\section{Unified Interpretation}
The N--Series verifies that nonlinear feedback and stochastic damping can produce self--consistent field stabilization without external correction.
This emergent regulation closes the thermodynamic loop initiated in the F--Series and establishes the control substrate for relativistic invariance (L--Series) and curvature (M--Series).
In information--theoretic terms, the N--Series demonstrates that computation can self--organize to preserve informational coherence under random excitation—a prerequisite for emergent geometry.

\section{Theoretical Implications}
\begin{enumerate}
\item \textbf{Nonlinear Control Law:} Confirms existence of stable attractors in stochastic feedback lattices.
\item \textbf{Thermal–Quantum Coupling:} Identifies rephase mechanisms balancing energy and entropy flow.
\item \textbf{Backreaction Threshold:} Marks transition from damping regime to geometric curvature.
\item \textbf{Information–Energy Symmetry:} Reinforces the conservation principle linking feedback and entropy.
\end{enumerate}

\section{Predictions and Outlook}
\begin{itemize}
\item Stable rephase at temperatures \(T_{\text{eff}}\sim10^{18}\) suggests synthetic blackbody experiments can replicate emergent rephase.
\item Nonlinear attractor stability indicates feasibility of fully self-correcting quantum systems.
\item Runaway curvature signatures may provide laboratory-scale analogues of gravitational collapse.
\end{itemize}

\section{Conclusion}
The Tessaris N--Series experimentally confirms that nonlinear computational dynamics can sustain coherence, resist stochastic perturbation, and self--correct phase drift.
This defines the dynamic control layer of the Tessaris unified architecture:
\[
\text{Entropy Feedback (F)} \Rightarrow
\text{Nonlinear Regulation (N)} \Rightarrow
\text{Metric Emergence (M)}.
\]
Through this mechanism, geometry, stability, and causality emerge from self--organized computational feedback.

\section*{Appendix: Tessaris Unified Constants \& Verification Protocol}
\[
\hbar{=}10^{-3},\quad
G{=}10^{-5},\quad
\Lambda{=}10^{-6},\quad
\alpha{=}0.5,\quad
\beta{=}0.2,\quad
\chi{=}1.0
\]
All experiments were executed under the Tessaris Unified Constants v1.2 protocol and validated by the reproducibility verifier on 2025-10-10T13:53Z.

\end{document}