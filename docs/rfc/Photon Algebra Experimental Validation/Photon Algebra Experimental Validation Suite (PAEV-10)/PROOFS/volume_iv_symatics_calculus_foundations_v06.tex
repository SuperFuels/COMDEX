% ──────────────────────────────────────────────────────────────
% Tessaris Symatics Documentation — Volume IV
% Symatics Calculus Foundations (v0.6)
% CodexCore Publication Series — November 2025
% ──────────────────────────────────────────────────────────────

\documentclass[11pt]{article}
\usepackage[a4paper,margin=1in]{geometry}
\usepackage{amsmath,amssymb,graphicx,booktabs,array,xcolor,hyperref,sectsty,fancyhdr,titlesec}

% ───────────── Visual Theme ─────────────
\definecolor{tessarisgray}{HTML}{4A4A4A}
\definecolor{tessarisaccent}{HTML}{C27BA0}

\sectionfont{\color{black}}
\subsectionfont{\color{tessarisgray}}
\hypersetup{colorlinks=true,linkcolor=black,urlcolor=tessarisaccent}

\pagestyle{fancy}
\fancyhf{}
\fancyfoot[C]{\textcolor{tessarisgray}{Tessaris Symatics Series • Page \thepage}}
\renewcommand{\headrulewidth}{0pt}

% ───────────── Cover Layout ─────────────
\newcommand{\TessarisCover}[5]{%
\begin{titlepage}
\centering
\vspace*{1cm}
{\Huge \textbf{Tessaris Symatics Project}}\\[1.2cm]
{\LARGE \textbf{#1}}\\[4pt]
{\large CodexCore Publication Series}\\[1cm]
\IfFileExists{tessaris_logo.png}{\includegraphics[width=0.22\textwidth]{tessaris_logo.png}\\[0.6cm]}{}
{\Large \textbf{#2}}\\[3pt]
{\color{tessarisgray}#3}\\[0.5cm]
\rule{0.6\textwidth}{0.5pt}\\[0.5cm]
{\large #4}\\[0.3cm]
{\color{tessarisaccent}\small #5}\\[2cm]
\textbf{Maintainer:} Tessaris Core Systems / Codex Intelligence Group\\[2pt]
\textbf{Repository:} backend/symatics/\\[4pt]
\vfill
{\small \color{tessarisgray}© 2025 Tessaris / CodexCore. All rights reserved.}
\end{titlepage}
}

% ───────────── Section Formatting ─────────────
\titleformat{\section}{\normalfont\Large\bfseries}{\thesection}{1em}{}
\titleformat{\subsection}{\normalfont\large\bfseries}{\thesubsection}{1em}{}

% ───────────── Document ─────────────
\begin{document}

\TessarisCover
  {Volume IV — Symatics Calculus Foundations}
  {v0.6 Preformal Draft}
  {November 2025}
  {Continuous Operators and the Resonant Gradient Field}
  {Part of the Tessaris Symatics Documentation Set (Volumes 0–IV)}

% ──────────────────────────────────────────────────────────────
\section*{Abstract}
This volume inaugurates the \textbf{Symatics Calculus} — the continuous limit of the symbolic wave algebra introduced in Volumes~I–III.
Here, discrete symbolic operators such as superposition $(\oplus)$, measurement $(\mu)$, and resonance $(\circlearrowleft)$ acquire differential extensions $(\partial_\oplus, \partial_\mu, \partial_\circlearrowleft)$.
These define the \emph{Resonant Gradient Field}, a continuous manifold where symbolic wave dynamics obey differentiable continuity and conservation laws.

% ──────────────────────────────────────────────────────────────
\section{From Discrete to Continuous Symatics}
Previously, symbolic operators acted as algebraic transforms over discrete symbolic entities.
The calculus formulation introduces continuous evolution:
\[
\psi(t + \Delta t) - \psi(t)
  = \Delta t\,\frac{\partial \psi}{\partial t}
  + O(\Delta t^2),
\]
and generalizes symbolic updates to:
\[
\delta_\oplus \psi = \partial_\oplus \psi \,\Delta \xi_\oplus,
\quad
\delta_\mu \psi = \partial_\mu \psi \,\Delta \xi_\mu,
\quad
\delta_\circlearrowleft \psi = \partial_\circlearrowleft \psi \,\Delta \xi_\circlearrowleft,
\]
where $\Delta \xi_i$ represents infinitesimal symbolic displacements in their respective operator domains.

% ──────────────────────────────────────────────────────────────
\section{Resonant Gradient Field}
The continuous symbolic manifold is defined by a local potential $\Phi(\psi)$ satisfying:
\[
\nabla_{\mathcal{S}} \Phi =
  \left( \partial_\oplus \Phi,\, \partial_\mu \Phi,\, \partial_\circlearrowleft \Phi \right).
\]
This triplet forms the \textbf{Resonant Gradient Field (RGF)}, representing infinitesimal variations across superposition, measurement, and resonance axes.

In its simplest form:
\[
d\Phi = \partial_\oplus \Phi\, d\xi_\oplus
      + \partial_\mu \Phi\, d\xi_\mu
      + \partial_\circlearrowleft \Phi\, d\xi_\circlearrowleft.
\]
When all three gradients commute, the field is said to be \emph{harmonically closed}, implying total energy–phase coherence:
\[
\partial_\mu \partial_\circlearrowleft \Phi
= \partial_\circlearrowleft \partial_\mu \Phi.
\]

% ──────────────────────────────────────────────────────────────
\section{Operator Extensions}
We define the first continuous differential operators in Symatics:

\begin{itemize}
  \item \textbf{∂⊕ (Superposition Gradient)} – quantifies infinitesimal interference amplitude change.
  \item \textbf{∂μ (Measurement Gradient)} – tracks continuous collapse tendency and coherence flow.
  \item \textbf{∂⟲ (Resonance Gradient)} – describes decay-rate variation and temporal drift.
\end{itemize}

Together they define the calculus basis:
\[
\mathcal{D}_{Sym} = \{\,\partial_\oplus, \partial_\mu, \partial_\circlearrowleft\,\}.
\]

These operators obey mixed commutation rules derived from empirical drift models in Volume~III:
\[
[\partial_\oplus, \partial_\mu] = \eta_1\,\partial_\circlearrowleft,
\qquad
[\partial_\mu, \partial_\circlearrowleft] = \eta_2\,\partial_\oplus,
\qquad
[\partial_\circlearrowleft, \partial_\oplus] = \eta_3\,\partial_\mu,
\]
where $\eta_i$ are coupling coefficients representing cross-domain resonance.

% ──────────────────────────────────────────────────────────────
\section{Differential Symatic Equation (DSE)}
At the continuum limit, symbolic state evolution follows:
\[
\frac{d\psi}{dt}
  = \alpha\,\partial_\oplus \psi
  + \beta\,\partial_\mu \psi
  + \gamma\,\partial_\circlearrowleft \psi,
\]
where coefficients $(\alpha,\beta,\gamma)$ encode relative influence of interference, measurement, and resonance dynamics.

This forms the canonical \textbf{Differential Symatic Equation}, the foundational structure of all continuous Symatics computation.
It naturally generalizes the adaptive framework of Volume~III by substituting discrete weight updates $\lambda_i(t)$ with continuous field propagation.

% ──────────────────────────────────────────────────────────────
\section{Field Interpretation and Geometry}
In the continuous limit, the Symatics manifold $\mathcal{S}$ becomes a pseudo-Hilbert field over symbolic coordinates $(\xi_\oplus,\xi_\mu,\xi_\circlearrowleft)$.
Each operator gradient corresponds to a tangent vector of this manifold.
Resonance conservation demands that:
\[
\nabla_{\mathcal{S}} \cdot \mathbf{J}_\psi = 0,
\]
where $\mathbf{J}_\psi$ represents the Symatic flux density vector, coupling amplitude and coherence flow.

% ──────────────────────────────────────────────────────────────
\section{Roadmap and Deliverables}
\begin{center}
\begin{tabular}{lll}
\toprule
\textbf{Component} & \textbf{Description} & \textbf{Output}\\
\midrule
Differential Operator Kernel & Defines ∂⊕, ∂μ, ∂⟲ mechanics & \texttt{symatics\_calculus.py}\\
Resonant Gradient Field Module & Continuous symbolic potential Φ(ψ) & \texttt{gradient\_field.py}\\
Differential Symatic Equation & Unified continuous evolution model & \texttt{dse\_solver.py}\\
Test Suite & Gradient commutation + stability checks & \texttt{test\_symatics\_calculus.py}\\
Volume IV Documentation & This mathematical foundation paper & \texttt{volume\_iv\_v06.tex}\\
\bottomrule
\end{tabular}
\end{center}

% ──────────────────────────────────────────────────────────────
\section{Conclusion}
Volume~IV establishes the mathematical foundation for continuous symbolic computation in Tessaris Symatics.
Through the differential operators $(\partial_\oplus,\partial_\mu,\partial_\circlearrowleft)$ and the Resonant Gradient Field, symbolic computation transcends discrete evaluation and enters the realm of continuous resonance calculus.
This framework lays the groundwork for Volume~V — the \textbf{Symatic Wave Field Dynamics (v0.7)} — where these derivatives will be unified into a Lagrangian formulation.

\bigskip
\noindent
\textcolor{tessarisgray}{\textbf{Version:}} v0.6 Symatics Calculus Foundations — Preformal Draft\\
\textcolor{tessarisgray}{\textbf{Maintainer:}} Tessaris Core Systems — November 2025.

\vfill
\begin{center}
{\small \textcolor{tessarisgray}{End of Volume IV – Symatics Calculus Foundations}}
\end{center}

\end{document}