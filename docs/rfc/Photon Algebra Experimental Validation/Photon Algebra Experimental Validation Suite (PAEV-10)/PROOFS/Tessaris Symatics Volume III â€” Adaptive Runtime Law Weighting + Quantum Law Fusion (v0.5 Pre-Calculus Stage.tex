

🧭 Phase I — Core Roadmap Outline (v0.5 Overview)

1. Core Concept — Adaptive Runtime Law Weighting

Each runtime law L_i acquires a dynamic coefficient \lambda_i(t) that evolves based on measured resonance drift, energy deviation, or symbolic asymmetry:

\frac{d\lambda_i}{dt} = \alpha_i \, \Delta E_i + \beta_i \, \Delta \varphi_i + \gamma_i \, \Delta \psi_i

These coefficients modulate validator thresholds and symbolic weighting within the evaluator.
The goal: self-tuning symbolic coherence — Tessaris adapts its internal law emphasis in response to quantum or energetic imbalance.

⸻

2. Mathematical Formalism — Weighted Law Superposition

Runtime validation becomes an adaptive superposition of laws:

\mathcal{L}_{runtime}(t) = \sum_i \lambda_i(t) \, L_i(\psi)

with normalization:
\sum_i \lambda_i(t) = 1

and coupling potentials introduced via resonant feedback:
\lambda_i(t+\Delta t) = \lambda_i(t) + \eta \, \nabla_\psi \, \mathcal{R}(\psi, t)

Here, \mathcal{R} denotes a resonance metric derived from phase-space coherence.
This introduces the first feedback derivative into Symatics — the proto-differential operator of the coming calculus.

⸻

3. Runtime Architecture Extension


The existing law_check system expands into an adaptive validator network:


# backend/symatics/core/validators/adaptive_laws.py

def update_law_weights(ctx, results):
    """
    Adjust λᵢ(t) coefficients based on recent law check results.
    """
    for law_id, outcome in results.items():
        drift = outcome.get("deviation", 0.0)
        λ = ctx.law_weights.get(law_id, 1.0)
        ctx.law_weights[law_id] = λ * (1 - α * drift)


	•	CodexTrace will record λᵢ(t) evolution per validation.
	•	The Tessaris backend maintains rolling averages of each drift metric for self-correction.

⸻

4. Fusion Domain — Quantum–Temporal Coupling

This is the conceptual fusion of three domains:

Domain
Operator
Physical Analog
Coupling
Measurement
μ / ∇
Collapse / Observation
μ ↔ ⟲
Temporal
⟲
Resonant Continuity
⟲ ↔ ↔
Quantum
↔
Entanglement / Correlation
μ–⟲–↔ Fusion


The resulting fusion law is expressed as:
\mathcal{F} = \mu \, \circlearrowleft \, \leftrightarrow
— representing a joint continuity between measurement, time, and correlation domains.

This will be the mathematical seed for Symatics Calculus Vol. III: the differential unification layer.

⸻

5. Deliverables (v0.5 Roadmap Summary)

Component
Description
Output
Adaptive Law Engine
Dynamic λᵢ(t) runtime model
adaptive_laws.py
Feedback Loop Integration
CodexTrace + law_check feedback
ctx.law_weights extension
Quantum–Temporal Fusion
μ–⟲–↔ coupling formalism
fusion_ops.py
Adaptive Test Set
Validation of λ dynamics & drift correction
test_adaptive_feedback.py
Volume III Doc
“Quantum Law Fusion Framework (v0.5 Draft)”
LaTeX + Markdown




📘 Structure Plan

I’ll produce two synchronized deliverables:
	1.	LaTeX master document — ready for Overleaf / TeX Live 2025 compilation (matches Volumes 0–II style).
	2.	Markdown companion draft — the readable spec summary for your CodexCore repo (docs/volume_iii_v05.md).

Both will include:
	•	Abstract & overview
	•	Formal equations for λᵢ(t) adaptive law coefficients
	•	Runtime architecture updates (adaptive_laws.py, ctx.law_weights)
	•	Fusion-domain coupling table (μ – ⟲ – ↔)
	•	Roadmap & deliverables list

⸻

📄 Step 1 — LaTeX Master Document

File: volume_iii_adaptive_fusion.tex

I’ll generate this next using your standard Tessaris front-matter style (gray text, no blue sections) so it visually aligns with Volumes I and II.

Once complete, you’ll have:
	•	A fully formatted publication PDF for v0.5 Adaptive Framework, and
	•	An automatically derived Markdown version for the documentation site.

⸻

Would you like me to include Appendix A: λ-Evolution Examples (plots or symbolic examples of adaptive weighting over time),
or keep this Volume III strictly textual like Volume II?

# Tessaris Symatics — Volume III  
### Adaptive Runtime Law Weighting + Quantum Law Fusion (v0.5)  
**CodexCore Publication Series — October 2025**

---

## Abstract
This volume establishes the **Adaptive Runtime Law Framework (ARLF)** of the Tessaris Symatics System.  
It extends the validated runtime layer of Volume II into a **dynamic adaptive regime**, introducing feedback-driven weighting of symbolic laws and the first instance of **quantum–temporal fusion**.  

Each symbolic constraint Lᵢ gains a dynamic coefficient λᵢ(t) that adapts in real time to resonance drift and energy deviation, bridging static algebraic invariance with continuous symbolic calculus.

---

## Overview
The Adaptive Runtime Law Framework defines a network of **self-adjusting symbolic validators**.  
Each runtime law contributes to the evaluation context as a weighted component:

\[
\mathcal{L}_{runtime}(t)=\sum_i \lambda_i(t)\,L_i(\psi),
\qquad
\sum_i \lambda_i(t)=1
\]

The λᵢ(t) evolve according to runtime feedback, transforming Symatics Algebra into a closed adaptive system capable of **self-correction and drift compensation**.

---

## Mathematical Formalism
Weight evolution is governed by deviations in key physical invariants:

\[
\frac{d\lambda_i}{dt}=\alpha_i\,\Delta E_i+\beta_i\,\Delta\varphi_i+\gamma_i\,\Delta\psi_i
\]

where ΔEᵢ is energy deviation, Δφᵢ phase shift, and Δψᵢ symbolic coherence error.  
Coefficients (αᵢ, βᵢ, γᵢ) control adaptive response rate.  

The verified runtime rule simplifies to:

\[
\lambda_i(t+\Delta t)=\lambda_i(t)\,[1-\alpha_i\,\Delta E_i]
\]

capturing **energy drift as the dominant stabilizing term** — the first temporal differential of Symatics, and the conceptual seed of its forthcoming Calculus layer.

---

## Runtime Architecture Extension
Version 0.5 augments the validator stack with adaptive feedback hooks.

```python
# backend/symatics/core/adaptive_laws.py
def update(self, law_id, deviation):
    current = self.weights.get(law_id, 1.0)
    updated = current * (1.0 - self.alpha * deviation)
    self.weights[law_id] = max(updated, 0.0)