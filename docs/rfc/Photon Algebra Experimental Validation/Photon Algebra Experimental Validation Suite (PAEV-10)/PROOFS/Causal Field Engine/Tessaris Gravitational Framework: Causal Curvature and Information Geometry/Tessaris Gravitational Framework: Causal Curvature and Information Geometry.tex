\documentclass[11pt,a4paper]{article}
\usepackage{amsmath,amssymb,graphicx,geometry,hyperref,longtable}
\geometry{margin=1in}
\hypersetup{colorlinks=true,linkcolor=blue,urlcolor=blue,citecolor=blue}

\title{\textbf{The Tessaris Gravitational Framework:\\Causal Curvature and Information Geometry}}
\author{Tessaris Research Group}
\date{October 2025}

\begin{document}
\maketitle

\begin{abstract}
The Tessaris Gravitational Framework redefines gravitation as a curvature of \emph{information geometry}, not of spacetime itself.  
Within the Tessaris Unified Constants environment, the $\Omega$--$\Xi$--X lattice tests demonstrated that large-scale energy distributions generate curvature in causal coherence, producing effects equivalent to gravity without invoking mass or metric tensors.  
Here we formalise that discovery: gravity arises from the \emph{curvature of causal flow}, an emergent property of self-organising information flux.
\end{abstract}

\section{1. Background}
In classical general relativity, the Einstein field equations relate curvature to energy density:
\[
G_{\mu\nu} = 8\pi G T_{\mu\nu}.
\]
However, this presumes spacetime as a fixed manifold and mass--energy as its source.  
In Tessaris, spacetime itself is an emergent lattice of information events governed by causal conservation:
\[
\nabla\!\cdot\!J_{\mathrm{info}} + \frac{\partial S}{\partial t} = 0.
\]
Curvature therefore arises when information flux ceases to propagate linearly, bending through the causal lattice.

\section{2. Core Discovery: Gravity as Information Curvature}
The $\Omega_2$ and $\Omega_3$ series revealed ``collapse'' and ``quantum bounce'' behaviour identical to gravitational confinement and rebound.  
From the unified data, Tessaris establishes:
\[
R_{\mathrm{causal}} \;\propto\; \nabla\!\cdot\!\nabla S_{\mathrm{info}},
\]
where $R_{\mathrm{causal}}$ is the curvature scalar of information space and $S_{\mathrm{info}}$ is local entropy density.  
Hence gravity is not the warping of spacetime by matter, but the bending of causal consistency by informational density.

\section{3. Causal Tensor Formulation}
We define the \emph{Causal Curvature Tensor}:
\[
\mathcal{C}_{\mu\nu} = \nabla_\mu J_{\nu}^{\mathrm{info}} - \nabla_\nu J_{\mu}^{\mathrm{info}}.
\]
The gravitational analogue arises as:
\[
R_{\mu\nu}^{\mathrm{causal}} = \partial_\mu \partial_\nu S_{\mathrm{info}} - g_{\mu\nu} \nabla^2 S_{\mathrm{info}}.
\]
The causal Einstein equation becomes:
\[
R_{\mu\nu}^{\mathrm{causal}} - \frac{1}{2} g_{\mu\nu} R^{\mathrm{causal}} = 8\pi \mathcal{G} J_{\mu\nu}^{\mathrm{info}},
\]
where $\mathcal{G}$ is the Tessaris causal coupling constant analogous to $G$.

\section{4. Physical Interpretation}
Curvature is the statistical average of causal deviation across the lattice.  
When local information flux slows or loops, the causal manifold contracts---analogous to gravitational potential.  
A steep entropy gradient (\(dS/dx\)) bends information trajectories, creating the perception of ``attraction.''

In equilibrium, information follows geodesics of minimal entropy production:
\[
\delta \int J_{\mathrm{info}} \cdot ds = 0.
\]

\section{5. Relation to Classical Gravity}
\begin{longtable}{|l|l|}
\hline
\textbf{General Relativity} & \textbf{Tessaris Causal Geometry} \\
\hline
Curvature of spacetime manifold & Curvature of causal information lattice \\
Energy--momentum tensor $T_{\mu\nu}$ & Information flux tensor $J_{\mu\nu}^{\mathrm{info}}$ \\
Geodesic motion of matter & Entropy-minimising causal trajectory \\
Mass--energy density & Information density $\rho_{\mathrm{info}}$ \\
Event horizon & Causal saturation boundary \\
\hline
\end{longtable}

Hence, gravity is reinterpreted as the large-scale coherence of information motion.  
The ``mass'' of an object corresponds to the degree of information compression it represents.

\section{6. Experimental Signatures}
\begin{enumerate}
\item \textbf{Quantum bounce (Ω₃):} partial recovery of collapsed information flux corresponds to gravitational rebound.  
\item \textbf{Optical curvature (Ξ₃):} light propagation through causal gradients mimics geodesic bending.  
\item \textbf{Thermal–quantum equilibrium (X₁):} balance residuals act as causal curvature scalars.  
\end{enumerate}
Together, these effects recreate gravitational dynamics purely from information evolution.

\section{7. Energy--Entropy Relationship}
From $\Omega$ and X series data:
\[
\langle |E| \rangle = 1.89\times10^{-1}, \qquad \langle |S| \rangle = 5.47\times10^{-3},
\]
yielding \(E/S \approx 34.6\).  
This ratio defines the curvature potential:
\[
\kappa_G = \frac{E}{S} \Rightarrow R_{\mathrm{causal}} \sim \kappa_G^{-1}.
\]
Regions of high informational energy relative to entropy are strongly curved---analogous to gravitational wells.

\section{8. Causal Geodesics}
Information follows paths of least causal action:
\[
\mathcal{A} = \int \rho_{\mathrm{info}} \sqrt{g_{\mu\nu} dx^\mu dx^\nu}.
\]
Variation yields:
\[
\frac{d^2 x^\mu}{d\tau^2} + \Gamma^\mu_{\nu\sigma} \frac{dx^\nu}{d\tau}\frac{dx^\sigma}{d\tau} = 0,
\]
where the connection coefficients $\Gamma^\mu_{\nu\sigma}$ are derived from information gradients, not metric deformation.  
Thus geodesics in Tessaris represent information-conserving trajectories.

\section{9. Gravitational Collapse and Recovery}
The $\Omega_1$--$\Omega_3$ sequence revealed that causal collapse (information saturation) is self-limiting.  
When $\nabla\!\cdot\!J_{\mathrm{info}} \to 0$, curvature diverges; when feedback from $\Xi$ synchrony reactivates, partial recovery occurs:
\[
\frac{\partial R_{\mathrm{causal}}}{\partial t} = -\gamma (\nabla\!\cdot\!J_{\mathrm{info}}) + \eta R_{\mathrm{sync}}.
\]
This expresses gravitational collapse and Hawking-like emission as a unified feedback process.

\section{10. Relation to Magnetism and Energy}
In Tessaris, magnetism and gravity are duals:
\[
\nabla\times J_{\mathrm{info}} \;\leftrightarrow\; \text{magnetic memory}, \quad
\nabla\!\cdot\!J_{\mathrm{info}} \;\leftrightarrow\; \text{gravitational curvature}.
\]
Magnetism stores direction (causal rotation), gravity stores cohesion (causal curvature).  
Together they maintain the integrity of the causal lattice.

\section{11. Engineering Implications}
\begin{enumerate}
\item \textbf{Causal curvature simulation:} gravitational behaviour can be emulated in QWaves by shaping information density.  
\item \textbf{Weightless computation:} local curvature cancellation allows energy-neutral field operation.  
\item \textbf{Photonic geodesics:} programmable curvature enables optical routing and energy focusing.  
\item \textbf{Containment design:} information curvature replaces pressure confinement---basis for low-energy fusion algorithms.  
\end{enumerate}

\section{12. Unified Causal Potential}
Define the scalar potential:
\[
\Phi_G = - \int R_{\mathrm{causal}} \, ds.
\]
Force emerges naturally:
\[
\mathbf{F}_G = -\nabla \Phi_G = -\nabla(\nabla^2 S_{\mathrm{info}}).
\]
Thus, gravitational attraction is the drive toward restoring uniform information density.

\section{13. Theoretical Implications}
Gravity is a thermodynamic and informational necessity:
\begin{itemize}
\item It prevents runaway entropy divergence by folding information trajectories.
\item It links energy localisation and synchrony across the lattice.
\item It guarantees finite coherence of computation in the physical domain.
\end{itemize}
In this sense, gravity is the \emph{causal integrity constraint} of the universe.

\section{14. Outlook}
The Tessaris Gravitational Framework completes the redefinition of the second fundamental field after magnetism.  
Future work includes:
\begin{enumerate}
\item Experimental validation of causal curvature via optical-lattice bending.  
\item Integration into the Photon Algebra as operator $\Phi_G$.  
\item Development of the \emph{Curvature Kernel} within QWaves for programmable gravity analogues.  
\item Application to cosmological simulations---Tessaris gravity as an alternative to dark matter curvature.  
\end{enumerate}

\section*{Summary}
\begin{longtable}{|l|l|l|}
\hline
\textbf{Quantity} & \textbf{Symbol / Value} & \textbf{Interpretation} \\
\hline
Information curvature scalar & $R_{\mathrm{causal}}$ & Curvature of causal lattice \\
Energy--entropy ratio & $E/S = 34.6$ & Defines gravitational potential \\
Causal coupling constant & $\mathcal{G}$ & Analogue of $G$ for information field \\
Collapse--recovery flux & $\langle J_{\mathrm{rec}}\rangle = 0.48$ & Quantum bounce equivalence \\
Synchrony factor & $R_{\mathrm{sync}}=0.995$ & Global coherence driver \\
\hline
\end{longtable}

\section{15. Conclusion}
Gravity, in the Tessaris model, is not a mysterious attractive force but the emergent curvature of information geometry.  
It arises whenever the causal flow of information becomes non-linear, producing the large-scale organisation and stability observed in physical systems.  
By expressing gravity as a programmable field of causal curvature, Tessaris bridges thermodynamics, computation, and spacetime into one coherent physical law.

\end{document}