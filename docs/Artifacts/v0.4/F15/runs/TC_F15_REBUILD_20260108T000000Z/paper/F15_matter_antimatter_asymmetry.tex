% -----------------------------------------------------------------------------
% Tessaris Truth Chain — v0.4 (F15) [LOCKABLE PAPER]
% Maintainer: Tessaris AI
% Author: Kevin Robinson
% Last update: 2026-01-08 (Europe/Madrid)
% -----------------------------------------------------------------------------
\documentclass[11pt,a4paper]{article}

% --- Tessaris / Symatics LaTeX house style ---
\usepackage[T1]{fontenc}
\usepackage[utf8]{inputenc}
\usepackage{newtxtext,newtxmath}
\usepackage{microtype}
\usepackage[a4paper,margin=1in]{geometry}

\usepackage{amsmath,amssymb}
\usepackage{booktabs}
\usepackage{tabularx}
\usepackage{array}
\usepackage{ragged2e}
\usepackage{enumitem}
\usepackage{xcolor}
\usepackage{xurl}
\usepackage[hidelinks]{hyperref}
\urlstyle{same}

\newcolumntype{Y}{>{\RaggedRight\arraybackslash}X}
\setlength{\emergencystretch}{3em}
\sloppy
\newcommand{\apath}[1]{\url{#1}}

\newcommand{\MUST}{\textbf{MUST}}
\newcommand{\SHOULD}{\textbf{SHOULD}}
\newcommand{\MAY}{\textbf{MAY}}

\title{\textbf{F15 — Matter--Antimatter Asymmetry}\\
\large Phase-Dominant CP Violation (model-scoped)}
\author{Maintainer: Tessaris AI \quad\textbullet\quad Author: Kevin Robinson}
\date{Pinned: 2026--01--08 (Europe/Madrid)}

\begin{document}
\maketitle

\begin{abstract}
This module tests whether a dual-field lattice $(\psi_1,\psi_2)$ in the Tessaris photon-algebra model
exhibits a persistent CP-like asymmetry under feedback evolution, using a small asymmetric injector
to probe symmetry stability. All terms are \textbf{model-scoped}. This paper is an indexable summary of the locked artifacts in this run.
\end{abstract}

\section*{Question}
Can the dual-field lattice $(\psi_1, \psi_2)$ within the Tessaris photon algebra exhibit spontaneous charge--parity (CP)
symmetry breaking — manifesting as an emergent matter--antimatter bias — without external forcing?

\section*{Method}
Two conjugate complex fields were initialized as:
\[
\psi_1, \psi_2 = e^{\pm i \phi(x,y)} e^{-((x\pm1.2)^2 + y^2)},
\]
with Gaussian curvature wells $\kappa(x,y)$ coupling to feedback-evolving parameters $\alpha(t)$ and $\Lambda(t)$.

A small CP-bias injector ($\Delta\Lambda = \pm 0.0025$) was applied asymmetrically between the two fields to probe symmetry stability.
The evolution equations were:
\[
\dot{\psi_i} = i\hbar\nabla^2\psi_i - \alpha_t \psi_i \pm i(\Lambda_t \pm \text{phase\_bias})\kappa \psi_i,
\]
integrated over 2400 steps with feedback coefficients $k_\alpha = 0.002$, $k_\Lambda = 0.0015$.

\section*{Measured signals}
\begin{itemize}[leftmargin=1.5em]
  \item Energy-density asymmetry $A(t) = (E_1 - E_2)/(E_1 + E_2)$
  \item Phase skew $\Delta\phi(t) = \arg\langle \psi_1 \psi_2^* \rangle$
  \item Mutual information proxy $I(t) = \langle \mathrm{Re}(\psi_1 \psi_2^*) \rangle$
\end{itemize}

\section*{Results (from locked summary JSON)}
\begin{itemize}[leftmargin=1.5em]
  \item Mean asymmetry: $A_{\text{tail}} = 2.28\times10^{-4}$
  \item Phase skew: $\Delta\phi_{\text{tail}} = 1.43~\text{rad}$
  \item Mutual information drift: $\Delta I = +987.3$
  \item Classification: \textbf{Phase-dominant CP violation (low-energy asymmetry)}
\end{itemize}

The mutual-information curve exhibited exponential amplification after timestep $\sim$1800, while energy asymmetry remained small but nonzero.
Phase skew stabilized near $\pi/2$, indicating a persistent parity offset between $\psi_1$ and $\psi_2$.

\section*{Interpretation}
Although energy densities between the dual fields remained closely matched, their phase correlation diverged irreversibly,
demonstrating a CP-like bias where informational coherence $I(t)$ dominates over energetic imbalance.

This emergent asymmetry is a model-analogue of CP violation in quantum fields: the system’s internal feedback loop selects a preferred phase basin,
breaking $\psi_1 \leftrightarrow \psi_2^*$ symmetry.

\section*{Significance}
This is a Tessaris photon-algebra test yielding stable phase-dominant CP-like bias under feedback evolution.
It provides a computational foundation for \emph{baryogenesis analogues} (model-scoped) — preference for “matter-like” states in a symmetric vacuum
— emerging from entanglement/feedback dynamics.

\section*{Artifacts (this run)}
\begin{itemize}[leftmargin=1.5em]
  \item Tests: \apath{../tests/paev_test_F15_matter_asymmetry.py}, \apath{../tests/paev_test_F15_entanglement_correlation.py}
  \item Knowledge: \apath{../knowledge/F15_matter_asymmetry.json}, \apath{../knowledge/F15_entanglement_correlation.json}
  \item Plots: \apath{../plots/PAEV_F15_Asymmetry.png}, \apath{../plots/PAEV_F15_PhaseSkew.png},
        \apath{../plots/PAEV_F15_MutualInfo.png}, \apath{../plots/PAEV_F15_DensityMaps.png},
        \apath{../plots/PAEV_F15_EntanglementCorrelation.png}
\end{itemize}

\section*{Figures}
\begin{figure}[h!]
  \centering
  \includegraphics[width=0.48\textwidth]{../plots/PAEV_F15_Asymmetry.png}
  \includegraphics[width=0.48\textwidth]{../plots/PAEV_F15_PhaseSkew.png}
  \caption{Energy-density asymmetry (left) and CP-like phase skew (right) showing stable phase offset.}
\end{figure}

\begin{figure}[h!]
  \centering
  \includegraphics[width=0.48\textwidth]{../plots/PAEV_F15_MutualInfo.png}
  \includegraphics[width=0.48\textwidth]{../plots/PAEV_F15_DensityMaps.png}
  \caption{Mutual information amplification (left) and final field amplitudes $|\psi_1|$, $|\psi_2|$ (right).}
\end{figure}

\begin{figure}[h!]
  \centering
  \includegraphics[width=0.70\textwidth]{../plots/PAEV_F15_EntanglementCorrelation.png}
  \caption{Entanglement correlation diagnostic (CHSH surrogate baseline; model-scoped).}
\end{figure}

% -----------------------------------------------------------------------------
% REQUIRED LOCK FOOTER FORMAT
% -----------------------------------------------------------------------------
\vspace{1.0em}
\noindent
Lock ID: \texttt{TC-F15-REBUILD-20260108T000000Z}\\
Status: \texttt{LOCKED}\\
Maintainer: Tessaris AI\\
Author: Kevin Robinson.

\end{document}
