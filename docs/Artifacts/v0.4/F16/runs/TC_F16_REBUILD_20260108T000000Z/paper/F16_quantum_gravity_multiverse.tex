% -----------------------------------------------------------------------------
% Tessaris Truth Chain — v0.4 (F16: Quantum Gravity and Multiverse Diversity) [LOCKED RUN PAPER]
% Maintainer: Tessaris AI
% Author: Kevin Robinson
% Pinned: 2026-01-08 (Europe/Madrid)
% -----------------------------------------------------------------------------
\documentclass[11pt,a4paper]{article}

% --- Tessaris / Symatics LaTeX house style ---
\usepackage[T1]{fontenc}
\usepackage{newtxtext,newtxmath}
\usepackage{microtype}
\usepackage[a4paper,margin=1in]{geometry}

\usepackage{amsmath,amssymb}
\usepackage{graphicx}
\usepackage{booktabs}
\usepackage{tabularx}
\usepackage{array}
\usepackage{ragged2e}
\usepackage{enumitem}
\usepackage{xurl}
\usepackage[hidelinks]{hyperref}
\urlstyle{same}

\newcolumntype{Y}{>{\RaggedRight\arraybackslash}X}
\setlength{\emergencystretch}{3em}
\sloppy

\newcommand{\apath}[1]{\url{#1}}
\newcommand{\MUST}{\textbf{MUST}}
\newcommand{\SHOULD}{\textbf{SHOULD}}
\newcommand{\MAY}{\textbf{MAY}}

\title{\textbf{F16 — Quantum Gravity and Multiverse Diversity}\\
\large Multidomain vacuum $\Lambda$ diversity under curvature feedback (model-scoped)}
\author{Maintainer: Tessaris AI \quad\textbullet\quad Author: Kevin Robinson}
\date{Locked run: TC\_F16\_REBUILD\_20260108T000000Z \; (Pinned: 2026--01--08, Europe/Madrid)}

\begin{document}
\maketitle

\begin{abstract}
F16 tests whether a quantum--gravitational feedback framework in the Tessaris photon algebra can
generate multiple self-consistent vacuum domains with distinct effective $\Lambda_i$ values,
forming a multiverse-like landscape of curvature equilibria \emph{within the model}.
This paper narrates the locked run outputs only; it makes no physical-world claims.
\end{abstract}

\section{Question (model-scoped)}
Can a quantum--gravitational feedback framework naturally generate multiple self-consistent vacuum domains
with distinct $\Lambda_i$ values, forming a multiverse-like landscape of curvature equilibria?

\section{Method}
The experiment evolves $N=64$ independent vacuum subdomains with randomized initial curvature offsets
$\Lambda_i(0)\in[0.5,1.5]\Lambda_0$. Each domain follows a curvature--feedback coupling law:
\[
\dot{\Lambda_i} = -\gamma_\Lambda (\Lambda_i - \Lambda_{\text{mean}}) + \xi_i(t),
\]
where $\xi_i(t)$ denotes small stochastic fluctuations (nominal $\sigma_\xi = 10^{-4}$).

Simulation parameters:
\[
(\hbar, G, \Lambda_0, \alpha) = (10^{-3}, 10^{-5}, 10^{-6}, 0.5), \quad
T = 2400,\ \Delta t = 0.006.
\]
Constants are verified against \texttt{constants\_v1.2.json} per the canonical test implementation.

\section{Results (locked run)}
The run reports a stable diversity of terminal $\Lambda_i$ values across domains with bounded curvature energy.
Classification: \textbf{✅ Multiverse-like $\Lambda$ diversity maintained} (model-scoped).

\section{Interpretation (model-scoped)}
The feedback terms preserve a spread of quasi-stable vacuum solutions without runaway divergence,
yielding a persistent ``landscape of equilibria in the model. This establishes the diversity leg
used by F17 (coupling/coherence) and later landscape convergence tests.

\section{Reproduce (canonical)}
\begin{verbatim}
export PYTHONPATH=.
python backend/photon_algebra/tests/paev_test_F16_quantum_gravity_multiverse.py
\end{verbatim}

\section{Artifacts}
\begin{itemize}[leftmargin=1.5em]
  \item Test (locked): \apath{docs/Artifacts/v0.4/F16/runs/TC_F16_REBUILD_20260108T000000Z/tests/paev_test_F16_quantum_gravity_multiverse.py}
  \item Summary JSON (locked): \apath{docs/Artifacts/v0.4/F16/runs/TC_F16_REBUILD_20260108T000000Z/knowledge/F16_quantum_gravity_multiverse.json}
  \item Plots (locked): \texttt{PAEV\_F16\_LambdaDiversity.png}, \texttt{PAEV\_F16\_SampleTraces.png}, \texttt{PAEV\_F16\_LandscapeMap.png}
\end{itemize}

\section{Figures}
\begin{figure}[h!]
  \centering
  \includegraphics[width=0.90\textwidth]{../plots/PAEV_F16_LambdaDiversity.png}
  \caption{$\Lambda_{\mathrm{eff}}$ diversity across domains (histogram).}
\end{figure}

\begin{figure}[h!]
  \centering
  \includegraphics[width=0.90\textwidth]{../plots/PAEV_F16_SampleTraces.png}
  \caption{Sample domain traces illustrating bounded evolution and persistent spread.}
\end{figure}

\begin{figure}[h!]
  \centering
  \includegraphics[width=0.90\textwidth]{../plots/PAEV_F16_LandscapeMap.png}
  \caption{$\kappa$-well landscape and domain distribution (model diagnostic visualization).}
\end{figure}

\vspace{1.0em}
\noindent
Lock ID: \texttt{TC-F16-REBUILD-20260108T000000Z}\\
Status: \texttt{LOCKED}\\
Maintainer: Tessaris AI\\
Author: Kevin Robinson.

\end{document}
