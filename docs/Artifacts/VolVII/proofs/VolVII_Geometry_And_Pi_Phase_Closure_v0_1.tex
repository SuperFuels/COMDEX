% Volume VII — Geometry and π: Phase Closure as the Origin of Form (v0.1)
\documentclass[11pt,a4paper]{article}

% --- Tessaris house style (Truth Chain compatible) ---
\usepackage[T1]{fontenc}
\usepackage{newtxtext,newtxmath}
\usepackage{microtype}
\usepackage{xurl}
\usepackage[hidelinks]{hyperref}
\usepackage[a4paper,margin=1in]{geometry}

% --- layout ---
\usepackage{tabularx}
\usepackage{ragged2e}
\usepackage{enumitem}
\usepackage{booktabs}

\newcolumntype{Y}{>{\RaggedRight\arraybackslash}X}

% --- overflow control (long paths / hashes / tt text) ---
\sloppy
\setlength{\emergencystretch}{2em}
\Urlmuskip=0mu plus 1mu\relax
\urlstyle{tt}
\newcommand{\codepath}[1]{\begingroup\ttfamily\small\path{#1}\endgroup}
\newcommand{\cmd}[1]{\begingroup\ttfamily\small\path{#1}\endgroup}

% --- Canonical glyph macros (chain-wide) ---
\newcommand{\superpose}{\oplus}
\newcommand{\entangle}{\leftrightarrow}
\newcommand{\recurse}{\circlearrowleft}
\newcommand{\measure}{\mu}     % measurement / collapse / selection
\newcommand{\project}{\pi}
\newcommand{\bornw}{\Delta}    % Born weight / intensity (NOT collapse)
\newcommand{\grad}{\nabla}     % RESERVED: geometric gradient/divergence only

\title{\textbf{Volume VII — Geometry and $\pi$: Phase Closure as the Origin of Form}\\
\large Deterministic Phase-Closure Evidence Surface (v0.1)}
\author{Kevin Robinson \\ \textit{Tessaris AI}}
\date{December 30, 2025}

\begin{document}
\maketitle

\noindent\textbf{Document ID:} VOLVII-GEOMETRY-PI-v0.1\\
\textbf{Status:} \textbf{DRAFT (pre-lock)}\\
\textbf{Maintainer:} Tessaris AI\\
\textbf{Author:} Kevin Robinson\\

\begin{abstract}
This volume formalizes \emph{phase closure} as the minimal mechanism by which stable geometric form emerges in Symatics.
The central invariant is the closure condition on a loop: $\oint d\phi = 2\pi_s n$.
A deterministic audit surface is specified: a byte-stable telemetry trace that (i) computes winding number, (ii) computes $\pi_s$, and (iii) demonstrates that the observed closure invariant matches Euclidean $\pi$ in the baseline reference scene.
This document is a verification contract and evidence spine for \emph{phase closure}, not a claim of new empirical physics.
\end{abstract}

\section{Definitions in Force (Truth Chain)}
\begin{itemize}[leftmargin=1.5em]
\item \textbf{Measurement / collapse / selection MUST use $\measure$.}
\item \textbf{$\bornw$ denotes intensity/weight (NOT collapse).}
\item \textbf{$\grad$ is RESERVED for geometric gradient/divergence only.}
\end{itemize}

\section{Phase Closure Invariant}
Let $\phi$ be a phase field on a closed loop $\ell$.
Phase closure is defined by:
\[
\oint_{\ell} d\phi = 2\pi_s n,\qquad n\in\mathbb{Z},
\]
where:
\begin{itemize}[leftmargin=1.5em]
\item $n$ is the winding number,
\item $\pi_s$ is the \emph{closure invariant} (baseline: $\pi_s=\pi$),
\item the \emph{closure error} is $\varepsilon_{\mathrm{cl}}=\left|\oint d\phi - 2\pi_s n\right|$.
\end{itemize}
In the reference scene, $\pi_s$ is computed as:
\[
\pi_s \;\triangleq\; \frac{1}{2n}\oint_{\ell} d\phi\quad (n\neq 0).
\]

\section{Discrete Ring Closure (Deterministic Reference Scene)}
The canonical reference scene instantiates a ring of $N$ unit complex phase carriers:
\[
z_k = e^{i\phi_k},\quad k\in\{0,\dots,N-1\},
\]
evolves deterministically by a local smoothing operator with renormalization back to unit magnitude, and emits an audit trace.
The closure integral is computed by unwrapped phase differences:
\[
\oint d\phi \;\approx\; \sum_{k=0}^{N-1}\mathrm{unwrap}(\phi_{k+1}-\phi_k),
\]
with $\phi_N \equiv \phi_0$ and $\mathrm{unwrap}(\cdot)$ mapping differences into $(-\pi,\pi]$.

\section{Telemetry Contract (Normative)}
All participants MUST emit:
\[
\texttt{record\_event}(event\_type,\; payload,\; context)
\]
with a JSON-serializable \texttt{payload} and \texttt{context}, and with top-level keys:
\[
\{\texttt{event\_type},\ \texttt{ts},\ \texttt{payload},\ \texttt{context}\}.
\]

\subsection{Required event types}
\begin{tabularx}{\linewidth}{@{}l Y Y@{}}
\toprule
\textbf{Event} & \textbf{Required fields} & \textbf{Purpose} \\
\midrule
\texttt{phase\_step} &
\texttt{step\_idx}, \texttt{dt}, \texttt{phi\_summary}, \texttt{coherence} &
Records per-step phase evolution summary. \\
\texttt{closure\_metric} &
\texttt{step\_idx}, \texttt{integral}, \texttt{winding\_n}, \texttt{closure\_error} &
Records loop integral + winding + closure error. \\
\texttt{pi\_estimate} &
\texttt{step\_idx}, \texttt{pi\_s}, \texttt{pi\_err}, \texttt{pi\_target} &
Records $\pi_s$ and deviation from $\pi$. \\
\bottomrule
\end{tabularx}

\subsection{Determinism requirement}
Given identical initial conditions, parameters, seed, and code version, the emitted JSONL trace MUST be byte-stable after canonical JSON normalization (sorted keys, stable separators).

\section{Verification Artifacts (Canonical Surface)}
This volume becomes lockable once the deterministic runner produces stable artifacts at:
\begin{itemize}[leftmargin=1.5em]
\item Scene spec: \codepath{docs/Artifacts/VolVII/qfc/VOL7_PHASE_CLOSURE_PI.scene.json}
\item Build manifest: \codepath{docs/Artifacts/VolVII/build/VOLVII_PIPELINE_MANIFEST.yaml}
\item Acceptance thresholds: \codepath{docs/Artifacts/VolVII/build/VOLVII_ACCEPTANCE_THRESHOLDS.yaml}
\item Evidence log: \codepath{docs/Artifacts/VolVII/ledger/VOL7_LINT_PROOF.log}
\item Metrics snapshot: \codepath{docs/Artifacts/VolVII/ledger/VOL7_METRICS.json}
\item Replay trace (JSONL): \codepath{docs/Artifacts/VolVII/ledger/VOL7_TRACE.jsonl}
\end{itemize}

\section{Executable Verification Evidence (Deterministic Runner)}
Runner:
\[
\cmd{python /workspaces/COMDEX/backend/tests/run\_vol7\_phase\_closure\_pi\_lint.py}
\]
It MUST:
\begin{itemize}[leftmargin=1.5em]
\item run the scene deterministically,
\item emit LOG/JSON/JSONL artifacts in Section 6,
\item enforce schema + required event types,
\item enforce byte-stable replay by running the scene twice and comparing trace digests.
\end{itemize}

\section{Recorded Passing Snapshot (Reference Scene)}
\noindent\textbf{Recorded passing snapshot (most recent run):}
\[
\texttt{steps=64}\quad
\texttt{dt=0.05}\quad
\texttt{seed=20251230}\quad
\texttt{events=192}
\]
\[
\texttt{winding\_n\_final=1}\quad
\texttt{closure\_error\_final=1.7763568394002505e{-15}}
\]
\[
\texttt{pi\_s\_final=3.1415926535897922}\quad
\texttt{pi\_err\_final=8.881784197001252e{-16}}\quad
\texttt{coherence\_final=0.9999667388144104}
\]
\[
\texttt{trace\_sha256=f7a5418c7200f0e7f1c41f58c3fb01d2ef7c60415cf36b47c48bcf36052e0b90}.
\]

\section{Lock Footer}
\noindent\rule{\linewidth}{0.4pt}

\noindent
Lock ID: VOLVII-GEOMETRY-PI-v0.1\\
Status: DRAFT (pre-lock)\\
Maintainer: Tessaris AI\\
Author: Kevin Robinson.

\end{document}
